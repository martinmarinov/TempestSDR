\newpage
{\Huge \bf Abstract}
\vspace{24pt} 


This dissertation presents a software toolkit for remotely eavesdropping video monitors using a Software Defined Radio (SDR) receiver. It exploits compromising emanations from cables carrying video signals. Analogue video is usually transmitted one line of pixels at a time encoded as a varying current. This generates a wideband electromagnetic wave that can be picked up by an SDR receiver. The presented software can map the received field strength of each pixel to a grayscale value in order to show a real-time false colour estimate of the original video signal.

The software significantly lowers the costs required for undertaking a practical attack compared to existing solutions. Furthermore, it allows for an additional digital post-processing which can aid in analysing and improving the results. It also provides mobility for a potential adversary, requiring only a commodity laptop and an USB SDR dongle. The attacker does not need to have any prior knowledge about the victim's video display. All parameters such as resolution and refresh rate can be estimated with the aid of the software. 

The software comprises of a library written in C, a collection of plug-ins for various Software Define Radio (SDR) frontends and a Java based Graphical User Interface (GUI). It is designed to be a multi-platform application. All native libraries can be pre-compiled and packed into a single Java jar file which allows the toolkit to run on any supported operating system.

This report documents the digital processing techniques that have been employed in order to extract, detect and lock to a video signal. It also explains the architecture of the software system and the techniques used in order to achieve low latency and real-time interactivity. It demonstrates the usage of the system by performing a practical attack. It then gives some ideas about what could be improved further and some analysis of data that was collected during the development of the software.


\newpage
\vspace*{\fill}
